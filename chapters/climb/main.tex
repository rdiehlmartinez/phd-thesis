\chapter{Curriculum Learning for Infant-inspired Model Building: A Framework for Human-like Language Acquisition}
\label{chapter:CLIMB}

While the background chapter established how human learning principles can inform language modeling, this chapter puts these ideas into practice through a concrete framework for curriculum learning. We present CLIMB (Curriculum Learning for Infant-inspired Model Building), a systematic approach to implementing developmentally plausible training protocols for language models. Our work is motivated by three key observations:

First, the current paradigm of language model training that relies on large datasets and computational resources stands in stark contrast to human language acquisition. Children acquire sophisticated language capabilities from only a few million words per year \citep{gilkerson2017mapping}, while state-of-the-art language models require trillions of tokens and extensive computational resources \citep{zhang2021need, zhao2023llmsurvey}. This discrepancy raises fundamental questions about the efficiency of current training approaches.

Second, conventional language model training differs from human learning in its structure: models operate on a predetermined static vocabulary and optimize a fixed objective on randomly shuffled data. In contrast, human language acquisition follows a carefully orchestrated progression—from babbling to simple utterances, and eventually to complex syntax and abstract meaning. This developmental trajectory suggests that structured learning protocols might enable more efficient model training.

Third, while curriculum learning has shown promise in various machine learning domains \citep{bengio2009curriculum}, its application to language modeling remains fragmented. Previous work has explored individual aspects such as vocabulary progression, data sequencing, or objective simplification but lacks a unified framework for implementing and evaluating these strategies, particularly in resource-constrained settings.

To address these challenges, we develop CLIMB within the context of the BabyLM Challenge \citep{warstadt2023babylm1}, which provides an ideal testbed for exploring cognitively plausible training protocols under strict data constraints (10 million words). Our framework systematically implements three curriculum dimensions:

\begin{itemize}
    \item \textbf{Vocabulary Curriculum:} Gradually expanding the model's lexicon, mirroring how children build their vocabulary from concrete nouns and verbs to more abstract terms.
    \item \textbf{Data Curriculum:} Structuring training data to progress from simpler to more complex linguistic structures, following the developmental trajectory observed in child language acquisition.
    \item \textbf{Objective Curriculum:} Evolving learning objectives from broad linguistic categories to specific token prediction, similar to how children first grasp word classes before mastering precise lexical distinctions.
\end{itemize}

Our work makes several key contributions:

\begin{enumerate}
    \item We establish a novel framework for categorizing and implementing curriculum learning strategies that simulate human language acquisition, providing a foundation for future research in cognitively inspired language modeling.
    
    \item Through extensive experimentation, we evaluate the effectiveness of different curriculum approaches under real-world constraints, offering concrete recommendations for when and how to apply specific curriculum strategies.
    
    \item We demonstrate that careful model and hyperparameter selection can yield strong performance even with limited data, with our vanilla models outperforming shared task baselines on grammatical knowledge (BLiMP) and approaching state-of-the-art performance on natural language understanding (SuperGLUE).
    
    \item We provide insights into the interaction between different curriculum dimensions, suggesting directions for developing more integrated approaches to curriculum learning in language modeling.
\end{enumerate}

The remainder of this chapter is organized as follows: Section 2 reviews the theoretical foundations of curriculum learning and its application to language modeling. Section 3 details our methodology, including the CLIMB framework and experimental setup. Section 4 presents our results and analysis, comparing different curriculum strategies and their combinations. Section 5 discusses the implications of our findings and directions for future work. Finally, Section 6 concludes with key insights and recommendations for implementing curriculum learning in language modeling.

\section{Methodology}

The work in this chapter is conducted within the context of the BabyLM Challenge \citep{warstadt2023babylm1}, which provides a constrained setting (10 million words) for exploring cognitively plausible training protocols. Before implementing our curriculum learning strategies, we first establish a strong baseline model and data processing pipeline. 

\subsection{Model Architecture and Training Setup}
All of our models are based on an 8-layer Transformer language model (\cref{subsec:baseline}) comparable to the BabyBERTa model \cite{huebner2021babyberta}. This architecture choice was motivated by the success of smaller models in low-resource settings, as demonstrated in the original BabyBERTa work. For all experiments, we leverage several key tools and frameworks to ensure robust and reproducible training. The Hugging Face Transformers library \cite{transformers} provides our model implementation, while Weights \& Biases \cite{wandb} enables comprehensive performance tracking and experiment monitoring. We use Hydra \cite{hydra} for experiment configuration management, allowing us to systematically explore different curriculum learning strategies. All training is conducted on a high performance computing cluster to ensure efficient model development and experimentation.

\subsection{Training Data}
\label{subsec:data}

\subsubsection{Data Source and Size}
We work exclusively with the training data provided in the \textsc{strict-small} track of the BabyLM challenge. This dataset is carefully constrained to 10 million words, compiled from 10 diverse corpora to ensure a representative sample of language use. Through our preprocessing pipeline, we reduce the initial 1,058,740 newline-delineated samples to 335,858 instances, corresponding to approximately 9.4 million words.\footnote{The word count is estimated by whitespace splitting, following the same metric used by the task organizers. When applying a tokenizer, the pre-processed dataset contains 11.7 million words (including punctuation) or 13.6 million subwords, reflecting the additional tokens introduced by subword tokenization.} This reduction in instances is primarily due to our concatenation strategy for shorter sequences, which we discuss in detail below.

\subsubsection{Data Preprocessing}
The diversity of our data sources (spanning books, subtitles, transcripts, and articles) necessitated careful curation to ensure consistency across corpora. Our preprocessing pipeline implements several key transformations. First, we standardize the text through lowercasing and punctuation normalization. We then apply regular expressions to standardize typographical conventions, ensuring consistent representation of numbers, dates, and special characters. The pipeline also removes extraneous content that could interfere with language modeling, including page numbers, bibliography entries, plain text tables, and one-word on-screen actions commonly found in subtitle data.

For transcribed speech corpora (with the exception of the British National Corpus), we implement a special concatenation strategy. Contiguous sections of five lines are combined into single data instances, addressing the challenge of relatively short sequence lengths in speech data. This approach helps maximize the effective use of our model's context window. Finally, at the point of model input, we join data segments to make full use of the available sequence length, which is set to 128 subtokens. This joining strategy is particularly important for maintaining the coherence of longer texts while staying within the model's context window constraints.

\subsubsection{Part-of-Speech Tagging}

While our preprocessing pipeline ensures consistent text formatting, we also need to capture linguistic structure to support our curriculum learning experiments. In particular, our vocabulary and objective curricula (Sections \ref{subsec:vocab-cl} and \ref{subsec:objective-cl}) rely on syntactic information to guide the learning process. However, the \textsc{strict-small} track's prohibition on external resources presented a unique challenge for implementing POS tagging, as we could not use supervised taggers.

\begin{wrapfigure}{r}{0.45\textwidth}
    \centering
    \small
    \begin{tabular}{lrrr}
    \toprule
    POS Tag & Precision & Recall & F1 \\
    \midrule
    NOUN & 0.786 & 0.790 & 0.788 \\
    DET & 0.820 & 0.772 & 0.795 \\
    CONJ & 0.969 & 0.821 & 0.895 \\
    NUM & 0.592 & 0.799 & 0.681 \\
    PRON & 0.592 & 0.962 & 0.733 \\   
    VERB & 0.816 & 0.823 & 0.819 \\
    PRT & 0.501 & 0.701 & 0.584 \\
    ADJ & 0.673 & 0.554 & 0.608 \\
    ADP & 0.842 & 0.888 & 0.864 \\
    PUNC & 0.944 & 0.960 & 0.952 \\
    \bottomrule
    \end{tabular}
    \caption{\label{tbl:unsupervised-pos-performance} Performance metrics of our unsupervised POS tagger compared to NLTK's supervised system.}
\end{wrapfigure}

To address this, we developed an unsupervised approach using the \texttt{anchor-features} algorithm \cite{stratos2016unsupervisedpos}, which identifies "anchor words" strongly associated with specific grammatical categories and uses these to learn a hidden Markov model (HMM). We ran this algorithm on the training dataset, and generated 30 clusters of features that each capture some latent syntactic information. We then manually mapped each cluster to a universal POS tag \cite{petrov2012universalpos}, with several clusters often mapping to the same grammatical category. Notably, our clustering approach failed to identify distinct groups for adverbs (ADV) and unknown tokens (X).

When evaluated against NLTK's supervised system \cite{bird2009natural}, our tagger showed strong performance on punctuation (F1: 0.952) and conjunctions (F1: 0.895), likely due to their consistent usage patterns. However, it struggled more with particles (F1: 0.584) and adjectives (F1: 0.608), which may have more variable usage patterns or semantic dependencies. These variations highlight the challenges of unsupervised grammatical category learning in low-resource settings.

\subsubsection{Data Availability and Observations}
To promote reproducibility and further research in this area, we provide our cleaned and tagged versions of the 10M word dataset on Hugging Face, along with the complete preprocessing scripts.\footnote{\url{https://huggingface.co/cambridge-climb}} This includes all the transformations described above, as well as the POS tagging pipeline. Interestingly, our experiments revealed that models trained on the raw, unprocessed data often outperformed those trained on our carefully preprocessed version. This counterintuitive finding, which we discuss in detail in Section~\ref{sec:discussion}, suggests that the linguistic "noise" in raw data may actually provide valuable learning signals for language models, particularly in low-resource settings.

\subsection{Vanilla Models}
\label{subsec:baseline}

\begin{table*}
\centering
\small
\setlength{\tabcolsep}{4pt}  % Reduce column spacing
\begin{tabular}{l | rrrrr | rrrr}
\toprule
Model  & L & H & Hidden & Int. & Vocab & Steps & BLiMP & BLiMP. Supp & Perplexity \\
\midrule
Small  & 8 & 8 & 256 & 2,048   & 8,192   & 250K      & 75.43      & 61.14       & 9.46    \\
Medium & 10 & 10 & 500 & 2,000 & 8,192  & 156K      & 76.45      & 63.28        & 9.05  \\
Large  & 12 & 12 & 768 & 3,072 & 8,192   & 94K      & 75.80      & 60.83      & 9.34 \\[2mm]
\hline \\
Small  & 8 & 8 & 256 & 2,048   & 16,384  & 250K      & 76.16      & 60.85       & 13.80    \\
Medium & 10 & 10 & 500 & 2,000  & 16,384 & 94K      & 76.09      & 60.03        & 13.80     \\
Large  & 12 & 12 & 768 & 3,072 & 16,384  & 62K      & 75.08      & 63.45      & 14.22     \\
\bottomrule
\end{tabular}
\caption{\label{tbl:baseline-size-comparison} Our vanilla BabyBERTa-style models evaluated on original BLiMP and the BLiMP-like tasks prepared for BabyLM (BLiMP.Supp). Models are grouped by their vocabulary sizes. L denotes the number of Transformer layers and H the number of attention heads per layer. The Hidden dimension (d\_model) represents the size of token representations at each layer, while Int. (Intermediate) indicates the expanded dimension size in the feed-forward network (typically 4x the hidden dimension).}
\end{table*}

We investigate three different sizes of a vanilla Pre-Layer Norm RoBERTa model \cite{liu2019roberta} based on the BabyBERTa model \cite{huebner2021babyberta}: `small', `medium', and `large' -- Table \ref{tbl:baseline-size-comparison} lists the model configurations and presents the results for the different model sizes evaluated by perplexity, on BLiMP \cite{warstadt2020blimp} and on the supplementary BLiMP-like tasks issued by the BabyLM organizers (`Blimp.Supp'). We found the medium model with a small vocabulary size performed the best overall; however, the small model achieved similar results, and so to save on compute and keep to the restrained intentions of the \textsc{strict-small} track, we used the small model in our curriculum learning experiments.
We use Byte Pair Encoding (BPE) tokenization \cite{gage1994bpe} with a vocabulary of 8,192 because it yields better overall performance compared to a larger vocabulary of 16,384. The tokenizers we use in our experiments were trained on the cleaned data that we processed using the steps outlined in \ref{subsec:data}. In pilot experiments, we did not observe the benefits reported by \citet{huebner2021babyberta} from removing the unmasking procedure that is a standard component of the MLM objective \cite{devlin2019bert}, and therefore did not investigate this option further.

All of the curriculum learning methods in the following sections were applied on top of our small vanilla BabyBERTa-style baseline -- to isolate the effect of the curriculum-learning training process, we fixed the architecture of the model and the model hyper-parameters. We use an AdamW optimizer with linear scheduling \cite{loshchilov2019decoupled}.

\section{A Three-Dimensional Framework for Curriculum Learning}
Curriculum learning \cite{bengio2009curriculum} is a machine-learning paradigm which optimizes a model's performance by gradually increasing the difficulty of training over time according to a set schedule (a `curriculum') -- based on the idea that learning should proceed from easy to hard, inspired by the way that humans learn \cite{elman1993learning}.
Within the context of curriculum learning, one of the central questions is how to define and manipulate the difficulty of the learning process over the course of training. In a recent survey, \citet{soviany2022curriculum} decompose this challenge into two main sub-problems: determining a sorting mechanism to assess the difficulty of instances and developing a pacing function for increasing difficulty over time. 

\subsection{Defining Difficulty across Three Dimensions}
Previous work in curriculum learning typically focuses on difficulty from a data-centric perspective, however, we note that difficulty can arise from (at least) three major elements of training a neural model: the input representation, the data sampling, and the training process. We explore curriculum learning strategies across three distinct dimensions: the vocabulary, the order of training data, and the objective function.

\paragraph{Vocabulary Difficulty} For machine learning models, instance difficulty is in part influenced by the choice of instance representation. For language models, the representational space is constrained by the vocabulary. We propose a new \textbf{vocabulary curriculum} inspired by \citet{soviany2022curriculum}, who discuss linking the curriculum criteria to the observed vocabulary sizes in child development. To the best of our knowledge, this is the first attempt at manipulating the vocabulary available to a language model through curriculum learning.

\paragraph{Data Difficulty} In natural language processing models, the order of the training instances can have a strong effect on performance \citep{schluter2018data}. 
Existing approaches to instance-level curriculum learning determine the difficulty of each instance according to a pre-defined static difficulty assessment according to linguistic criteria \citep{campos2021curriculum,kocmi2017curriculum,liu2018curriculum,platanios2019competence}. It has been shown that humans pay more attention to stimuli that are in just the right zone of difficulty for them: neither too easy nor too hard \cite{kidd2012goldilocks}. This so-called `Goldilocks effect' can be modelled by assessing the difficulty of an instance dynamically based on model behaviour \citep{sachan2016easy,lalor2020dynamic}. Static and dynamic difficulty assessment can be mapped to teacher-centric and learner-centric educational approaches and we compare both variants in our \textbf{data curriculum} experiments. 

\paragraph{Objective Difficulty} Human language learning is guided and enabled to some extent by other agents in the learner's environment (e.g.,\ adult caregivers, siblings) who interact with the learner. In machine learning, such interactions are modelled by the objective function that guides the weight optimization process. The typical `masked language modelling' (MLM) objective function requires that a model predicts a target token from a pre-defined vocabulary of size $N$ given the surrounding context. Thus standard MLM defines an $N$-way token classification task.

Curriculum learning can be leveraged within this context to attenuate the difficulty of the classification task during training. One natural starting point for doing so is to redefine the classification task to be over a smaller set of items, $K$, such that $ K << N$. \citet{bai2022better} map rare words with hypernyms of that word to simplify the classification task in training. A related line of research suggests replacing certain words with either part-of-speech tags \cite{wang2023language} or syntactic dependency relations \cite{cui2022lert}. Since the number of syntactic tags is substantially smaller than the number of vocabulary items, these approaches greatly reduce the difficulty of the objective. Moreover, by varying the amount of syntactic tags that the model should classify over, the difficulty of the task can be dynamically adapted \cite{wang2023language}. We take inspiration from this latter line of work in defining our own \textbf{objective curriculum}. 

\subsection{Pacing Functions} Once a notion of difficulty is set, a pacing function is needed to govern how quickly the model will progress from training on easier examples to training on harder ones \cite{wu2021when}. We experiment with two different pacing functions: linear and logarithmic. Linear pacing functions involve a steady and consistent advancement through the curriculum. This approach ensures a gradual increase in difficulty over time. Logarithmic pacing functions, on the other hand, emphasize early exposure to ``easier'' concepts, with diminishing increments as the model's capabilities are assumed to increase. Both pacing functions have been proposed in the broader curriculum learning literature \citep{bai2022better, li2021curriculum, wu2021when}. In \cref{fig:pacing_fn} we illustrate the two pacing functions.


\begin{figure}
    \centering
    \includegraphics[width=.70\textwidth]{chapters/climb/figures/pacing_fns.png}
    \caption{Illustration of the linear and logarithmic pacing functions used in our vocabulary curriculum experiments. The red dotted lines denote the curriculum regime, during which the percentage of unmasked words available to the model grows according to the respective function.}
    \label{fig:pacing_fn}
\end{figure}

\begin{table*}
\vspace{-10mm}
\centering
\small
\begin{tabular}{lll}
\toprule
\textbf{Curriculum Type} & \textbf{Parameter} &\textbf{Variants} \\
\midrule
 \multirow{2}{*}{Vocabulary} & Selection & frequency, word class, mixed \\
 & Pacing & linear, logarithmic \\
 \midrule
 \multirow{3}{*}{Data} & Difficulty & source, unigram perplexity, self-perplexity \\
 & Pacing & linear, logarithmic \\
 & Initial Perplexity & unigram, random \\
  \midrule
 \multirow{2}{*}{Objective} & Tasks & noun-verb prediction, POS prediction, MLM\\
 & Learning Setup & sequential, multitask \\
\bottomrule
\end{tabular}
\caption{\label{tbl:configurations} Curriculum learning experiments overview}
\end{table*}


\subsection{Vocabulary Curriculum}
\label{subsec:vocab-cl}

During the early stages of language acquisition, children start with a small vocabulary that rapidly expands at a rate of eight to ten words per day \cite{weizman2001lexical}. In this process, children prioritize learning verbs and nouns before progressing to other parts of speech \cite{bergelson2015early}. Large language models, on the other hand, tend to begin training with a full, fixed vocabulary available to them. 

To represent a child's growing vocabulary, we select a limited vocabulary in the initial stages of learning and map all other input tokens into the representation for the unknown token (\textsc{UNK}). We consider three strategies for selecting tokens. In the first strategy, tokens are selected according to frequency. We approximate the frequency of a token by the identifier the BPE tokenizer assigns to it as lower IDs are assigned to tokens that are merged first (i.e., sequences of characters that occur more frequently in the corpus). In the second strategy, tokens are selected by their word class. We approximate the word class of a token by the cluster that the unsupervised POS-tagger assigns to it. We order the word classes as follows, progressing from lexical to functional classes per \citet{bergelson2015early}: NOUN, VERB, ADJ, PRON, DET, ADP, NUM, CONJ, PRT, PNCT. In this strategy, all words with the respective part-of-speech tag are included in the vocabulary at the same step during learning. To smooth this process, we combine the frequency and the word class constraint in the third strategy.
We sort words by their frequency (approximated by the token ID) within each part-of-speech category. Note that the same word may be available in some instances and not others if it is assigned a more difficult POS tag. 

During the initial steps of training, only 10\% of the tokens are available while the rest are replaced with UNK. The vocabulary curriculum regime begins after 25,000 training steps and ends at 350,000 steps, during which time, the vocabulary gradually increases according to a pacing function. We experiment with linear and logarithmic pacing functions.  After the end of the curriculum regime, there remain 50,000 training steps before the end of training during which all of the vocabulary tokens are available to the model. Figure \ref{fig:pacing_fn} in the Appendix shows a plot of the percentage of unmasked vocabulary over the course of training according to our pacing functions.


\subsection{Data Curriculum}
\label{subsec:data-cl}

Conventional masked language modeling approaches train a given neural network on a large amount of crawled internet data. The resulting text sequences are usually not curated beyond basic cleaning and are presented to the model in random order, in contrast to the way that human children learn a language.
 

\paragraph{Source Difficulty} We order the available datasets based on their sources so that spoken samples are considered `easier' and purely written texts `harder', following the findings of \citet{huebner2021babyberta}. Within this ordering, we place the mostly child-directed speech from CHILDES before adult-to-adult dialogues in the Switchboard Corpus, and Simple Wikipedia before Wikipedia, see Table \ref{tbl:source_order}.\footnote{There is likely some adult-to-adult dialogue included in CHILDES as well.}

\begin{table}
\centering
\small
\begin{tabular}{cl}
\toprule
Difficulty Level & Corpora \\
\midrule
1 & AO-CHILDES\\
2 & BNC Spoken, Switchboard \\
3 & Open Subtitles, QED \\
4 & CBT, Children's Stories \\
5 & Simple Wikipedia \\
6 & Wikipedia, Gutenberg \\
\bottomrule
\end{tabular}
\caption{\label{tbl:source_order} Difficulty level assigned to each dataset.}
\end{table}

\paragraph{Model Difficulty}
Determining the difficulty of an instance based on its data source is a relatively naive heuristic that ignores the variation of instance difficulty within one corpus. As a more fine-grained alternative, we determine the difficulty of each instance individually using the model-intrinsic metric of perplexity which determines the likelihood of a sentence. We experiment with two variants: a static unigram language model and a more dynamic self-evaluation. With the unigram model, perplexity for each instance is only determined once at the beginning of training. Alternatively, we evaluate the perplexity of the remaining training data using the model that has been trained so far -- from model checkpoints saved at regular intervals in training (every 25K steps). 

One challenge with the latter approach is the lack of exposure to training data at the beginning, leading to random perplexity scores for each sample. To address this, we propose two ideas: 1) using a separately trained unigram model to initially evaluate perplexity, or 2) initially sample training instances randomly. After 25,000 training steps, we switch to using the current model for perplexity evaluation. Every 25,000 steps thereafter, we re-evaluate perplexity to identify samples categorized as relatively difficult or relatively easy by the model.

\subsection{Objective Curriculum}
\label{subsec:objective-cl}

The MLM objective has proven tremendously successful in training Transformer networks as language models \cite{devlin2019bert}. Psycholinguistic research, however, suggests that MLM is not a cognitively plausible approximation of language acquisition processes in children \cite{caucheteux2023evidence}.
Curriculum learning establishes a framework for varying the difficulty of the learning process over the course of training. The MLM objective is a very challenging discriminative classification task because the identity of the masked token needs to be determined over the entire vocabulary. We experiment with using more coarse-grained tasks at the initial stages of training to facilitate generalization and leverage syntactic information. Research in cognitive linguistics has shown that one-year-old infants are sensitive to distributional aspects of language and from two years of age begin to recognize lexical categories such as nouns and verbs \citet{alishahi2010computational, gleitman1990structural}. We therefore experiment with predicting only the word class of a masked token at the start of training rather than predicting its exact target token ID. 

The psycholinguistic literature remains divided on the question of how exactly word learning proceeds from memorizing a small set of fixed lexical items to a more generalized representation of word classes \cite{clark2015first}. Our framework provides a flexible approach to vary the difficulty of objective functions during the course of training, and to enable systematic studies of the effect of objective functions on the acquisition of linguistic knowledge by a model.
Here we propose estimating the word class using the unsupervised POS tagger and we vary the number of POS tags which are being classified over. The masked word is classified into 1) one of VERB, NOUN, or OTHER, or 2) one of 10 universal POS tags. 

We examine activating the tasks in sequential order (first word class prediction then MLM) or optimizing them in parallel, comparable to a multi-task learning setting. For each objective function, we learn a separate task head with its own linear task classifier and separate optimizer. 



\begin{figure*}
\centering
\includegraphics[height=9.5cm]{chapters/climb/figures/babylm_blimp_diffs_boxplots.png}
\caption{\label{fig:blimp-boxplots} Comparison of the BabyLM baselines with our BabyBERTa-style vanilla models (left), and our vanilla models against our curriculum learning models (right) -- using BabyBERTa-small trained on clean data as a reference point (asterisked) to show the difference in scores on BLiMP and BLiMP-supplement tasks. For combination models, all pacing is logarithmic, and `multitask' refers to the 2-task objective curriculum, 10 POS-tags and MLM from the outset. Absolute values may be found in Appendix Tables~\ref{tbl:result-vocab-cl}--\ref{tbl:submission-comparison}.
}
\end{figure*}

\begin{figure*}
    \centering
    \includegraphics[height=9.5cm]{chapters/climb/figures/babylm_superglue_diffs_boxplots.png}
    \caption{\label{fig:glue-boxplots} Comparison of the BabyLM baselines with our BabyBERTa-style vanilla models (left), and our vanilla models against our curriculum learning models (right) -- using BabyBERTa-small trained on clean data as a reference point (asterisked) to show the difference in scores on SuperGLUE tasks. For combination models, all pacing is logarithmic, and `multitask' refers to the 2-task objective curriculum, 10 POS-tags and MLM from the outset. 
    }
\end{figure*}

\section{Results}

Multiple evaluation metrics are employed in BabyLM. In this paper we focus on BLiMP \cite{warstadt2020blimp} and the supplementary BLiMP-style tests provided by the shared task organizers. We also report our results on the natural language understanding benchmark, SuperGLUE \cite{wang2019superglue}, and the ambiguous subset of MSGS (the Mixed Signals Generalization Set) \cite{warstadt2020msgs}. In brief, BLiMP evaluates specific linguistic abilities, MSGS evaluates linguistic preference over surface generalisation and SuperGLUE evaluates downstream task performance. For all scores, we report the average score across all categories, rather than test instances, as provided by the BabyLM evaluation pipeline.\footnote{For instance, there are 12 categories in BLiMP but 50+ individual tests. We average over the scores given for each category, rather than the scores given for each test.} All of our curriculum learning models are small BabyBERTa-style ones using the parameters shown in Table~\ref{tbl:baseline-size-comparison} and the cleaned training dataset of 9.4M words (reduced from the 10M word dataset for the \textsc{strict-small} track) and their results can be found in Tables \ref{tbl:result-vocab-cl}, \ref{tbl:result-data-cl} and \ref{tbl:result-obj-cl}. 

In the tables we compare to our small BabyBERTa-style vanilla model also trained on the clean data (Section \ref{subsec:baseline}). Figure~\ref{fig:blimp-boxplots} visualizes these comparisons for the BLiMP tasks; there are  similar plots for SuperGLUE in the Appendix (Figure~\ref{fig:glue-boxplots}).

Furthermore, we experimented with some combinations of different curricula to see how they would interact (Table~\ref{tbl:result-combination-cl}), and compare the official BabyLM shared-task baselines with our shared task entries -- a number of our own BabyBERTa-style vanilla models and curriculum learning models (Table~\ref{tbl:submission-comparison}). For all of our runs, we use the same set of hyper-parameters that we report in Table~\ref{tbl:baseline_hyperparams}. We also report the average amount of compute used for each type of curriculum learning setting (Table~\ref{tbl:compute}).

\begin{table*}
    \centering
    \small
    \begin{tabular}{llc}
    \toprule
         Type & Model & Training Time \\
    \midrule
         Vanilla Models & CLIMB-small-raw & 12h \\
         & CLIMB-raw (medium) & 17h40m \\
    \midrule
         Data Curriculum & Log Source & 12h30m \\
         & Log Random + model ppl & 17h10m \\
         Objective Curriculum & Sequential All POS & 11h40m \\
         & Multitask All POS & 15h30m \\
         Vocabulary Curriculum & Linear POS & 11h50m \\
         & Log Token ID & 12h10m \\
    \midrule
        Combination & Log Data Split + Log Token ID & 12h30m \\
        & Log Random + model ppl + Log Token ID & 17h10m \\
    \bottomrule
    \end{tabular}
    \caption{Compute required to train our models. We report the model with the shortest and longest runtime for each experiment type. Each model is trained for 400,000 steps with 4 A100 GPUs.}
    \label{tbl:compute}
\end{table*}

We find notable gains for our own vanilla models over the shared-task baselines, and, while we do not identify further large improvements in our curriculum learning models, we do notice some modest gains which suggest possibilities for future research and experimentation over variables. While the differences in performance between most of our experimental conditions are small, the large number of ablations we run enables us to provide a comprehensive set of recommendations for how and when different curriculum learning strategies may offer improved performance on linguistic tasks.
Below we summarize our observations over the full results tables.


\paragraph{In general, log pacing works at least as well as linear pacing across different curricula learning strategies.}
In our data curriculum experiments, models using the log pacing function outperform their linear counterparts in 4/4 settings on BLiMP, and 3/4 settings for BLiMP-supplement and SuperGLUE (Table~\ref{tbl:result-data-cl}). This indicates that rapidly increasing the difficulty of training instances in the early stages brings downstream benefits on grammaticality and NLU tasks.

In our vocabulary curriculum experiments on the other hand, there is not such a clear picture. Log pacing outperforms linear in 2/3 settings on BLiMP and 3/3 on SuperGLUE, but 0/3 for BLiMP-supplement (Table~\ref{tbl:result-vocab-cl}).
Presumably this is a reflection of the different vocabulary required by each set of evaluation tasks, which could be a matter for future investigation but also indicates that we do not yet have a clear generalizable pacing function for the vocabulary curriculum. There are of course other pacing functions to be tried.

\paragraph{Different representations of vocabulary difficulty work better for different tasks.}
When representing difficulty in the vocabulary curriculum experiments, token ID -- our proxy for frequency -- appears to work better than word classes (POS tags) or a combination of token ID and POS tags on the BLiMP evaluation tasks, but worse than POS tags on SuperGLUE and MSGS (Table~\ref{tbl:result-vocab-cl}).


\paragraph{In multi-corpora datasets, ordering by difficulty is a good first step.}
Training data requirements have grown so much in modern NLP that usually training a language model from scratch will involve multiple datasets, or multiple domains. The results of our data curriculum experiments indicate that a good first step is to put these sub-corpora into some order of intuitive difficulty, as we did (Table~\ref{tbl:result-data-cl}). In the case of BLiMP this approach outperforms our perplexity-based data curricula, and with log pacing our vanilla model. The same is true of MSGS  (with log pacing), as well as BLiMP-supplement and SuperGLUE (though the last two do not beat our vanilla model). 
Amongst the perplexity-driven models, the picture is less positive: out of 24 tests, only one model outperforms our vanilla model (log pacing, random initialisation + model perplexity in Table~\ref{tbl:result-data-cl}).

\paragraph{Multitask learning holds sway over sequentially swapping objective functions for now.}
In our experiments with curricula for the objective function, we compare training on simultaneous tasks -- known as multitask learning \cite{caruana1997multitask} -- with predefined sequences of objective functions which swap from one to another at set thresholds in the training process. We set up two sequential curricula: one with 2 tasks (predicting the 10 universal POS tags found in our dataset, and MLM) and the other with 3 (like the 2 task curriculum, additionally with noun/verb/other prediction). We compare these against multitasking alternatives. In general the sequential curricula are outperformed by the multitasking ones, though the 3-task sequential curriculum outperforms our BabyBERTa-style vanilla model on SuperGLUE and is second only marginally to our best-performing multitask model (Table~\ref{tbl:result-obj-cl}). The multitask learning model with 10-class universal POS-tag prediction and MLM in place from the outset performs best on BLiMP and SuperGLUE. However, our best model on BLiMP-supplement -- a multitask one -- has an element of sequential task scheduling in that the two POS-tag prediction tasks are lined up one after the other, with a switch from 3-class to 10-class after 6.25\% of training steps. In Figure \ref{fig:baseline_obj_cl_blimp_supp}, we visualize this result for each task in BLiMP-supplement, illustrating that our curriculum learning model improves over our vanilla model in 5/6 tasks.
Altogether, these results suggest that sequential objective function curricula do hold some potential for performance gains if further tuning of the tasks and scheduling can be carried out.

\paragraph{Combining all three curricula shows potential on BLiMP.}
While each individual curriculum learning experiment did not result in consistent improvements across tasks, we investigated whether combining aspects from the different curricula would, together, improve the model.
We do find that a combination of all three curricula outperforms any single curriculum model on BLiMP, but the same is not true for BLiMP-supplement and SuperGLUE (Table~\ref{tbl:result-combination-cl}). This is another matter for future investigation, as it seems that improving each of the three curricula we investigate may lead to further gains if they are all combined.


\paragraph{In small data settings, filtering data which we intuitively think is noisy is in fact counter-productive.} Perhaps surprisingly, we find that the vanilla models trained on the raw data outperform those trained on the pre-processed data on BLiMP and MSGS. We surmise that models can learn even from linguistically non-standard datapoints.

\section{Discussion}\label{sec:discussion}

We set out to investigate a number of curriculum learning approaches to language model training, motivated by findings from the human language acquisition process and by the wish to successfully train smaller models for smaller budgets.
We first of all implemented a stronger model of our own, based on BabyBERTa \cite{huebner2021babyberta} and found that a small 8-layer vanilla model could outperform the provided BabyLM baselines on the BLiMP grammaticality tests and get close to the best RoBERTa shared-task baseline on SuperGLUE. This underlines the findings reported in the BabyBERTa paper: that with smaller datasets, it makes sense to use smaller models and a smaller vocabulary size.

The results of our curriculum learning experiments, trained with a small BabyBERTa-style vanilla model, suggest that we can further improve performance in certain linguistic tasks by careful application of a pacing function, how we represent and grow the model's vocabulary during training, select the next training instances according to their difficulty, and vary the objective function. Specifically, we find that a logarithmic pacing function works better for the data curriculum than a linear one, but the findings for the vocabulary curriculum are less clear. Other pacing functions might be tried in the future, including those that reflect acquisition theory around non-monotonic or `U-shaped' development trajectories.

It is apparent that ordering the subcorpora within a training set may be worthwhile, and that perplexity-based approaches to data selection hold potential even though we have not found a clear-cut best method for perplexity calculation as yet. As shown in other NLP work, multitask learning can be a beneficial approach, though MLM or next-word prediction remain preeminent as singular tasks used in language modeling. We find multitask learning models hard to beat in the objective curriculum, but do find good performance in our sequential settings. We believe that future work varying the timing of task switches and introducing more tasks could be worthwhile.

On a more general note, the Baby LM challenge evaluates a language model only on its final downstream performance on a set of tasks -- i.e.\ at a finite point in time. The challenge does not directly measure whether a given model is learning in a `human-like' fashion. Our contribution to the BabyLM challenge is to provide a set of curriculum learning strategies which are motivated by the language learning dynamics of infants and children. We encourage future research to study how to quantitatively evaluate whether the learning trajectory of a model parallels that of a human language learner and how similarities to human language learning results in downstream NLU performance. 

\section{Conclusions}
We use child-like language learning as inspiration to investigate and implement three types of curriculum learning for language modelling: gradually increasing the size of the vocabulary (\textbf{vocabulary curriculum}), the difficulty of the training instances (\textbf{data curriculum}), or the specificity of the objective function (\textbf{objective curriculum}).

We find that our BabyBERTa-style vanilla models outperform the BabyLM baselines on BLiMP and MSGS, and get close on SuperGLUE. Our various curriculum learning models at times offer further gains over our vanilla models, and indicate the potential for curriculum learning methods given further exploration. We list out a set of recommendations for when and how to optimally apply our proposed curriculum learning strategies.

Additionally, training our vanilla model trained on unprocessed data outperforms a `cleaned' version -- suggesting that retaining as much data as possible, in low-resource settings, is more important than standardizing it according to linguistic norms.

Finally, our work establishes a computational framework for how to categorize and implement curricula learning strategies that simulate human language learning dynamics. 

\paragraph{Objective curriculum models on BLiMP Supplement and (Super)GLUE.} Figures \ref{fig:baseline_obj_cl_blimp_supp} and \ref{fig:baseline_obj_cl_superglue} compare our small BabyBERTa-style vanilla  model to our best objective curriculum learning setting -- a multi-task trained model with sequential POS-tag prediction -- on each task in BLiMP Supplement and (Super)GLUE. We find our curriculum-learning (CL) model outperforms our vanilla model on 5/6 tasks in BLiMP Supplement. While on (Super)GLUE, our CL model outperforms our baseline on 4/10 tasks and obtains comparable performance on another 4/10 tasks. This results illustrate the potential to further explore objective-curricula settings.

\begin{figure}[h]
\centering
\includegraphics[width=0.5 \textwidth]{chapters/climb/figures/baseline_vs_obj_cl_blimp_supp_new.png}
\caption{\label{fig:baseline_obj_cl_blimp_supp} Comparison between our vanilla model and the best objective curriculum learning setting on the BLiMP supplementary tasks.}
\end{figure}

\begin{figure}[h]
\centering
\includegraphics[width=0.5 \textwidth]{chapters/climb/figures/baseline_vs_obj_cl_superglue.png}
\caption{\label{fig:baseline_obj_cl_superglue} Comparison between our vanilla model and the best objective curriculum learning setting on the (Super)GLUE tasks.}
\end{figure}


\begin{table*}
\centering
\small
\begin{tabular}{llrrrrr}
\toprule
Pacing & Difficulty         & Perplexity & BLiMP & BLiMP.Supp & (Super)GLUE & MSGS Ambig \\
\midrule
\textsuperscript{\textdagger}Linear & Token ID      &     9.70& 75.09 & \textbf{66.43}      & 68.71            & 68.61    \\
Linear & POS                &   10.17& 72.06 & 63.44      & 69.50            & 66.91    \\
Linear & POS + Token ID     &   10.21& 73.37 & 66.11      & 69.22            & 66.61    \\
Log    & Token ID           &   9.26& 74.97 & 64.63      & 69.94            & 66.82    \\
Log    & POS                &  9.29& 74.12 & 62.06      & \textbf{70.66}            & \textbf{70.52}    \\
Log    & POS + Token ID     &  9.29& 74.74 & 63.62      & 70.29            & 66.42    \\
\midrule
Vanilla Model &  &  \textbf{9.21} &  \textbf{75.48} & 65.34 & 70.47 & 68.30 \\
\bottomrule
\end{tabular}
\caption{\label{tbl:result-vocab-cl} Results for vocabulary curriculum models (Section \ref{subsec:vocab-cl}). All models score above 90 in the MSGS Control tasks. \textsuperscript{\textdagger} indicates the model we submitted to BabyLM, `CLIMB-tokens'. }
\end{table*}


\begin{table*}
\centering
\small
\begin{tabular}{llrrrrr}
\toprule
Pacing & Difficulty         & Perplexity & BLiMP & BLiMP.Supp & (Super)GLUE & MSGS Ambig \\
\midrule
Linear & Source         &    10.41& 73.32 & 61.99      & 69.68      & 66.22    \\
Linear & Unigram ppx     &       12.51& 72.45 & 61.67      & 69.10           & 66.90    \\
Linear & Unigram + model ppx &  11.88& 72.62 & 62.57      & 69.86           & 66.64    \\
Linear & Random + model ppx  &  10.82& 71.88 & 63.10      & 70.37            & 67.48    \\
\textsuperscript{\textdagger}Log    & Source       &      \textbf{9.21}& \textbf{75.87} & 64.29      & 70.20           & \textbf{70.99}    \\
Log    & Unigram ppx      &     9.39& 75.03 & 63.78      & 69.90            & 66.69    \\
Log    & Unigram + model ppx &  9.35& 74.83 & 64.24      & 70.09           & 66.89    \\
Log    & Random + model ppx  &  \textbf{9.21}& 75.81 & 63.03      & 68.93         & 66.64    \\
\midrule
Vanilla Model & &\textbf{9.21}  & 75.48 & \textbf{65.34} & \textbf{70.47} & 68.30 \\
\bottomrule
\end{tabular}
\caption{\label{tbl:result-data-cl} Results for data curriculum models (Section \ref{subsec:data-cl}). All models score above 92 in the MSGS Control tasks. \textsuperscript{\textdagger} indicates the model we submitted to BabyLM, `CLIMB-data-split'. }
\end{table*}


\begin{table*}
\centering
\small
\begin{tabular}{l lll | rrrrr}
\toprule
 & \multicolumn{3}{c}{Task duration (\% of training steps)} & & & & \\
Task Order & 3 POS                                & 10 POS      & MLM        & PPX & BLiMP & BLiMP.Supp & (Super)GLUE & MSGS Ambig \\
\midrule
Sequential       & --       & 0 - 12.5    & 12.5 - 100 & 9.58  & 73.87 & 62.98      & 69.85       & 66.70    \\
Multitask      & --       & 0 - 100     & 12.5 - 100 & 9.78   & 74.60 & 62.17     & 69.12       & 66.64    \\
Multitask      & --       & 0 - 100     & 0 - 100    & 9.30  & \textbf{75.82} & 65.77     & \textbf{70.74}       & 66.58    \\
Sequential       & 0 - 6.25 & 6.25 - 12.5 & 12.5 - 100 & 9.49  & 74.03 & 63.02      & 70.71       & 66.93    \\
Multitask      & 0 - 6.25 & 6.25 - 100  & 12.5 - 100 & 9.72  & 73.68 & 63.89     & 70.07       & 67.00    \\
\textsuperscript{\textdagger}Multitask      & 0 - 6.25 & 6.25 - 100  & 0 - 100    &  9.30 & 74.80 & \textbf{67.55}      & 69.89       & 67.65    \\
Multitask      & 0 - 100  & --          & 0 - 100   & 9.25  & 74.48 & 63.98     & 69.77       & 67.72    \\
\midrule
Vanilla Model & &  & & \textbf{9.21}  & 75.48 & 65.34 & 70.47 & \textbf{68.30} \\
\bottomrule
\end{tabular}
\caption{\label{tbl:result-obj-cl} Results for objective curriculum models (Section \ref{subsec:objective-cl}). All models score above 94 in the MSGS Control tasks. Task duration defines when an objective function was active during training, as a percentage of the total number of training steps. \textsuperscript{\textdagger} indicates the model we submitted to BabyLM, `CLIMB-multitask'. }
\end{table*}


\begin{table*}
\centering
\small
\begin{tabular}{lll|rrrrr}
\toprule
Vocab Curric.\ & Data Curric.\ & Obj. Curric.\ & PPX & BLiMP & BLiMP.Supp & (Super)GLUE & MSGS Ambig  \\
\midrule
-- & Source & Multitask &                           9.29& 74.06 & 64.06 & 70.02 & 66.90 \\
-- & Random + model ppx & Multitask &                9.44& 75.89 & 64.63 & 69.72 & 67.78 \\
Token ID & Source & -- &                     
 9.27& 75.89 & 64.62 & 70.24 & 67.90 \\
Token ID & Random + model ppx & -- &
    9.30& 75.88 & \textbf{65.79} & 70.42 & 66.63 \\
Token ID & Source & Multitask &
     9.22 & 74.86 & 62.82 & 70.09 & 66.68 \\
Token ID & Random + model ppx & Multitask & 9.46& \textbf{75.92} & 63.68 & 69.98 & \textbf{71.30} \\
\midrule
Vanilla Model & & & \textbf{9.21} & 75.48 & 65.34 & \textbf{70.47} & 68.30 \\
\bottomrule
\end{tabular}
\caption{\label{tbl:result-combination-cl} Results for the combination curriculum models. The multitask objective curriculum refers to the 2-task 10-POS and MLM model shown in Table~\ref{tbl:result-obj-cl}. }
\end{table*}


\begin{table*}
\centering
\small
\begin{tabular}{llrrrrr}
\toprule
Type              & Model    & PPX   & BLiMP & BLiMP.Supp & (Super)GLUE & MSGS Ambig \\
\midrule
Official Baseline & OPT-125m         & --    &   63.16 & 55.08 & 63.38 & 69.22 \\
                  & RoBERTa-base      & --  &   
                69.84 & 50.52 & 71.42 & 70.25 \\
                  & T5-base         & --    &   58.27 & 47.55  & 60.93 & 68.55 \\
\midrule
Vanilla Models    &CLIMB-base (medium)   & 9.01   & 75.66 & 66.13 & 70.75 & 67.62 \\
                  & CLIMB-base-small & 9.21  & 75.48 & 65.34 & 70.47 & 68.30 \\
                  & CLIMB-raw (medium)   &  8.47   & 77.97 & 66.16 & 70.63 & 69.44 \\
                  & CLIMB-small-raw  & 8.64  & 76.42 & 64.60 & 69.46 & 70.65 \\
                & \emph{large-100M}      & 4.35      &   81.03 & 75.56 & 72.93 & 74.17 \\
\midrule
Vocab Curriculum          & CLIMB-tokens   &  9.70  & 75.09 & 66.43  & 68.71 & 68.61 \\
Data Curriculum           & CLIMB-data-split & 9.21 & 75.87 & 64.29  & 70.20 & 70.99 \\
Objective Curriculum      & CLIMB-multitask & 9.30 & 74.80 & 67.55  & 69.89 & 67.65 \\
\bottomrule
\end{tabular}
\caption{\label{tbl:submission-comparison} Comparison between the official shared task baselines, our BabyBERTa-style vanilla models, and our submitted curriculum learning models on the main evaluation tasks: BLiMP, (Super)GLUE, and MSGS. Our *small and *medium models are defined in Section \ref{subsec:baseline}. All models are trained on pre-processed data except for those labelled with *-raw, which are trained on mostly unprocessed data (except we join the input sentences). The `large-100M' model was a larger BabyBERTa-style model trained on the 100M BabyLM training set (all others have been trained on the 10M dataset available in the \textsc{strict-small} track). }
\end{table*}

\begin{table*}
    \centering
    \small
    \begin{tabular}{lc}
    \toprule
         Parameter& Value\\
    \midrule
         Layer Norm EPS& 1e-5 \\
         Tie Word Embeddings & False \\
         Learning Rate & 0.001 \\
         Optimizer & AdamW \\
         Scheduler Type & Linear\\
         Max Steps & 400,000 \\
         Warm-up Steps & 100,000\\
         Per Device Batch Size & 32 \\
    \bottomrule
    \end{tabular}
    \caption{Hyperparameter settings which are constant across our vanilla models described in \ref{subsec:baseline}. Table \ref{tbl:baseline-size-comparison} reports variations to the architectures to create the `small', `medium' and `large' versions of the vanilla model. Where values are not reported, they may be assumed to be default values.}
    \label{tbl:baseline_hyperparams}
\end{table*}

\begin{table*}
    \centering
    \small
    \begin{tabular}{llc}
    \toprule
         Type & Model & Training Time \\
    \midrule
         Vanilla Models & CLIMB-small-raw & 12h \\
         & CLIMB-raw (medium) & 17h40m \\
    \midrule
         Data Curriculum & Log Source & 12h30m \\
         & Log Random + model ppl & 17h10m \\
         Objective Curriculum & Sequential All POS & 11h40m \\
         & Multitask All POS & 15h30m \\
         Vocabulary Curriculum & Linear POS & 11h50m \\
         & Log Token ID & 12h10m \\
    \midrule
        Combination & Log Data Split + Log Token ID & 12h30m \\
        & Log Random + model ppl + Log Token ID & 17h10m \\
    \bottomrule
    \end{tabular}
    \caption{Compute required to train our models. We report the model with the shortest and longest runtime for each experiment type. Each model is trained for 400,000 steps with 4 A100 GPUs.}
    \label{tbl:compute}
\end{table*}


% --- Additional Stuff ---

% \subsection{Curriculum Learning Approach}
% Our work introduces curriculum learning to three primary components of language model pre-training: vocabulary (Section \ref{subsec:vocab-cl}), data sampling approach (Section \ref{subsec:data-cl}), and objective function selection (Section \ref{subsec:objective-cl}). This multi-faceted approach is inspired by how humans learn language, where different aspects of language acquisition develop simultaneously but at different rates. For each component, we implement dynamic difficulty scaling that evolves over the course of training, simulating the progressive nature of human language learning. The specific variables and configurations for each curriculum learning experiment are detailed in Table~\ref{tbl:configurations}.